%
% API Documentation for API Documentation
% Module pycosmicstar.cosmology
%
% Generated by epydoc 3.0.1
% [Mon Aug 11 14:45:22 2014]
%

%%%%%%%%%%%%%%%%%%%%%%%%%%%%%%%%%%%%%%%%%%%%%%%%%%%%%%%%%%%%%%%%%%%%%%%%%%%
%%                          Module Description                           %%
%%%%%%%%%%%%%%%%%%%%%%%%%%%%%%%%%%%%%%%%%%%%%%%%%%%%%%%%%%%%%%%%%%%%%%%%%%%

    \index{pycosmicstar \textit{(package)}!pycosmicstar.cosmology \textit{(module)}|(}
\section{Module pycosmicstar.cosmology}

    \label{pycosmicstar:cosmology}
\textbf{Version:} 1.0.1



\textbf{Author:} Eduardo dos Santos Pereira



\textbf{License:} GPLV3




%%%%%%%%%%%%%%%%%%%%%%%%%%%%%%%%%%%%%%%%%%%%%%%%%%%%%%%%%%%%%%%%%%%%%%%%%%%
%%                               Variables                               %%
%%%%%%%%%%%%%%%%%%%%%%%%%%%%%%%%%%%%%%%%%%%%%%%%%%%%%%%%%%%%%%%%%%%%%%%%%%%

  \subsection{Variables}

    \vspace{-1cm}
\hspace{\varindent}\begin{longtable}{|p{\varnamewidth}|p{\vardescrwidth}|l}
\cline{1-2}
\cline{1-2} \centering \textbf{Name} & \centering \textbf{Description}& \\
\cline{1-2}
\endhead\cline{1-2}\multicolumn{3}{r}{\small\textit{continued on next page}}\\\endfoot\cline{1-2}
\endlastfoot\raggedright \_\-\_\-e\-m\-a\-i\-l\-\_\-\_\- & \raggedright \textbf{Value:} 
{\tt \texttt{'}\texttt{pereira.somoza@gmail.com}\texttt{'}}&\\
\cline{1-2}
\raggedright \_\-\_\-c\-r\-e\-d\-i\-t\-s\-\_\-\_\- & \raggedright \textbf{Value:} 
{\tt \texttt{[}\texttt{'}\texttt{Eduardo dos Santos Pereira}\texttt{'}\texttt{]}}&\\
\cline{1-2}
\raggedright \_\-\_\-m\-a\-i\-n\-t\-a\-i\-n\-e\-r\-\_\-\_\- & \raggedright \textbf{Value:} 
{\tt \texttt{'}\texttt{Eduardo dos Santos Pereira}\texttt{'}}&\\
\cline{1-2}
\raggedright \_\-\_\-s\-t\-a\-t\-u\-s\-\_\-\_\- & \raggedright Abstract Class of cosmological models.

          This file is part of pystar. copyright : Eduardo dos Santos 
          Pereira

          pystar is free software: you can redistribute it and/or modify it
          under the terms of the GNU General Public License as published by
          the Free Software Foundation, either version 3 of the License. 
          pystar is distributed in the hope that it will be useful, but 
          WITHOUT ANY WARRANTY; without even the implied warranty of 
          MERCHANTABILITY or FITNESS FOR A PARTICULAR PURPOSE.  See the GNU
          General Public License for more details.

          You should have received a copy of the GNU General Public License
          along with Foobar.  If not, see 
          {\textless}http://www.gnu.org/licenses/{\textgreater}.

\textbf{Value:} 
{\tt \texttt{'}\texttt{Stable}\texttt{'}}&\\
\cline{1-2}
\raggedright \_\-\_\-p\-a\-c\-k\-a\-g\-e\-\_\-\_\- & \raggedright \textbf{Value:} 
{\tt None}&\\
\cline{1-2}
\end{longtable}


%%%%%%%%%%%%%%%%%%%%%%%%%%%%%%%%%%%%%%%%%%%%%%%%%%%%%%%%%%%%%%%%%%%%%%%%%%%
%%                           Class Description                           %%
%%%%%%%%%%%%%%%%%%%%%%%%%%%%%%%%%%%%%%%%%%%%%%%%%%%%%%%%%%%%%%%%%%%%%%%%%%%

    \index{pycosmicstar \textit{(package)}!pycosmicstar.cosmology \textit{(module)}!pycosmicstar.cosmology.cosmology \textit{(class)}|(}
\subsection{Class cosmology}

    \label{pycosmicstar:cosmology:cosmology}
\textbf{Known Subclasses:} pycosmicstar.lcdmcosmology.lcdmcosmology


%%%%%%%%%%%%%%%%%%%%%%%%%%%%%%%%%%%%%%%%%%%%%%%%%%%%%%%%%%%%%%%%%%%%%%%%%%%
%%                                Methods                                %%
%%%%%%%%%%%%%%%%%%%%%%%%%%%%%%%%%%%%%%%%%%%%%%%%%%%%%%%%%%%%%%%%%%%%%%%%%%%

  \subsubsection{Methods}

    \label{pycosmicstar:cosmology:cosmology:dt_dz}
    \index{pycosmicstar \textit{(package)}!pycosmicstar.cosmology \textit{(module)}!pycosmicstar.cosmology.cosmology \textit{(class)}!pycosmicstar.cosmology.cosmology.dt\_dz \textit{(method)}}

    \vspace{0.5ex}

\hspace{.8\funcindent}\begin{boxedminipage}{\funcwidth}

    \raggedright \textbf{dt\_dz}(\textit{self}, \textit{z})

    \vspace{-1.5ex}

    \rule{\textwidth}{0.5\fboxrule}
\setlength{\parskip}{2ex}
    Return the relation between the cosmic time and the redshift

\setlength{\parskip}{1ex}
    \end{boxedminipage}

    \label{pycosmicstar:cosmology:cosmology:dr_dz}
    \index{pycosmicstar \textit{(package)}!pycosmicstar.cosmology \textit{(module)}!pycosmicstar.cosmology.cosmology \textit{(class)}!pycosmicstar.cosmology.cosmology.dr\_dz \textit{(method)}}

    \vspace{0.5ex}

\hspace{.8\funcindent}\begin{boxedminipage}{\funcwidth}

    \raggedright \textbf{dr\_dz}(\textit{self}, \textit{z})

    \vspace{-1.5ex}

    \rule{\textwidth}{0.5\fboxrule}
\setlength{\parskip}{2ex}
    Return the comove-radii

\setlength{\parskip}{1ex}
    \end{boxedminipage}

    \label{pycosmicstar:cosmology:cosmology:dV_dz}
    \index{pycosmicstar \textit{(package)}!pycosmicstar.cosmology \textit{(module)}!pycosmicstar.cosmology.cosmology \textit{(class)}!pycosmicstar.cosmology.cosmology.dV\_dz \textit{(method)}}

    \vspace{0.5ex}

\hspace{.8\funcindent}\begin{boxedminipage}{\funcwidth}

    \raggedright \textbf{dV\_dz}(\textit{self}, \textit{z})

    \vspace{-1.5ex}

    \rule{\textwidth}{0.5\fboxrule}
\setlength{\parskip}{2ex}
    Return the comove volume

\setlength{\parskip}{1ex}
    \end{boxedminipage}

    \label{pycosmicstar:cosmology:cosmology:rodm}
    \index{pycosmicstar \textit{(package)}!pycosmicstar.cosmology \textit{(module)}!pycosmicstar.cosmology.cosmology \textit{(class)}!pycosmicstar.cosmology.cosmology.rodm \textit{(method)}}

    \vspace{0.5ex}

\hspace{.8\funcindent}\begin{boxedminipage}{\funcwidth}

    \raggedright \textbf{rodm}(\textit{self}, \textit{z})

    \vspace{-1.5ex}

    \rule{\textwidth}{0.5\fboxrule}
\setlength{\parskip}{2ex}
    Return the Dark Matter Density

\setlength{\parskip}{1ex}
    \end{boxedminipage}

    \label{pycosmicstar:cosmology:cosmology:robr}
    \index{pycosmicstar \textit{(package)}!pycosmicstar.cosmology \textit{(module)}!pycosmicstar.cosmology.cosmology \textit{(class)}!pycosmicstar.cosmology.cosmology.robr \textit{(method)}}

    \vspace{0.5ex}

\hspace{.8\funcindent}\begin{boxedminipage}{\funcwidth}

    \raggedright \textbf{robr}(\textit{self}, \textit{z})

    \vspace{-1.5ex}

    \rule{\textwidth}{0.5\fboxrule}
\setlength{\parskip}{2ex}
    Return the Barionic Matter Density

\setlength{\parskip}{1ex}
    \end{boxedminipage}

    \label{pycosmicstar:cosmology:cosmology:H}
    \index{pycosmicstar \textit{(package)}!pycosmicstar.cosmology \textit{(module)}!pycosmicstar.cosmology.cosmology \textit{(class)}!pycosmicstar.cosmology.cosmology.H \textit{(method)}}

    \vspace{0.5ex}

\hspace{.8\funcindent}\begin{boxedminipage}{\funcwidth}

    \raggedright \textbf{H}(\textit{self}, \textit{z})

    \vspace{-1.5ex}

    \rule{\textwidth}{0.5\fboxrule}
\setlength{\parskip}{2ex}
    Return the Hubble Parameter

\setlength{\parskip}{1ex}
    \end{boxedminipage}

    \label{pycosmicstar:cosmology:cosmology:dgrowth_dt}
    \index{pycosmicstar \textit{(package)}!pycosmicstar.cosmology \textit{(module)}!pycosmicstar.cosmology.cosmology \textit{(class)}!pycosmicstar.cosmology.cosmology.dgrowth\_dt \textit{(method)}}

    \vspace{0.5ex}

\hspace{.8\funcindent}\begin{boxedminipage}{\funcwidth}

    \raggedright \textbf{dgrowth\_dt}(\textit{self}, \textit{z})

    \vspace{-1.5ex}

    \rule{\textwidth}{0.5\fboxrule}
\setlength{\parskip}{2ex}
    Return the derivative of growth function of the primordial 
    perturbations

\setlength{\parskip}{1ex}
    \end{boxedminipage}

    \label{pycosmicstar:cosmology:cosmology:growthFunction}
    \index{pycosmicstar \textit{(package)}!pycosmicstar.cosmology \textit{(module)}!pycosmicstar.cosmology.cosmology \textit{(class)}!pycosmicstar.cosmology.cosmology.growthFunction \textit{(method)}}

    \vspace{0.5ex}

\hspace{.8\funcindent}\begin{boxedminipage}{\funcwidth}

    \raggedright \textbf{growthFunction}(\textit{self}, \textit{z})

    \vspace{-1.5ex}

    \rule{\textwidth}{0.5\fboxrule}
\setlength{\parskip}{2ex}
    Return the growth function of the primordial perturbations

\setlength{\parskip}{1ex}
    \end{boxedminipage}

    \label{pycosmicstar:cosmology:cosmology:dsigma2_dk}
    \index{pycosmicstar \textit{(package)}!pycosmicstar.cosmology \textit{(module)}!pycosmicstar.cosmology.cosmology \textit{(class)}!pycosmicstar.cosmology.cosmology.dsigma2\_dk \textit{(method)}}

    \vspace{0.5ex}

\hspace{.8\funcindent}\begin{boxedminipage}{\funcwidth}

    \raggedright \textbf{dsigma2\_dk}(\textit{self}, \textit{kl})

    \vspace{-1.5ex}

    \rule{\textwidth}{0.5\fboxrule}
\setlength{\parskip}{2ex}
    "Return the integrating of sigma(M,z) for a top-hat filtering. In z = 0
    return sigma\_8, for z {\textgreater} 0 return sigma(M,z)

\setlength{\parskip}{1ex}
    \end{boxedminipage}

    \label{pycosmicstar:cosmology:cosmology:sigma}
    \index{pycosmicstar \textit{(package)}!pycosmicstar.cosmology \textit{(module)}!pycosmicstar.cosmology.cosmology \textit{(class)}!pycosmicstar.cosmology.cosmology.sigma \textit{(method)}}

    \vspace{0.5ex}

\hspace{.8\funcindent}\begin{boxedminipage}{\funcwidth}

    \raggedright \textbf{sigma}(\textit{self})

    \vspace{-1.5ex}

    \rule{\textwidth}{0.5\fboxrule}
\setlength{\parskip}{2ex}
    Return  the variance of the linear density field. As pointed out by 
    Jenkis et al. (2001), this definition of the mass function has the 
    advantage that it does not explicitly depend on redshift, power 
    spectrum or cosmology.

\setlength{\parskip}{1ex}
    \end{boxedminipage}

    \label{pycosmicstar:cosmology:cosmology:age}
    \index{pycosmicstar \textit{(package)}!pycosmicstar.cosmology \textit{(module)}!pycosmicstar.cosmology.cosmology \textit{(class)}!pycosmicstar.cosmology.cosmology.age \textit{(method)}}

    \vspace{0.5ex}

\hspace{.8\funcindent}\begin{boxedminipage}{\funcwidth}

    \raggedright \textbf{age}(\textit{self}, \textit{z})

    \vspace{-1.5ex}

    \rule{\textwidth}{0.5\fboxrule}
\setlength{\parskip}{2ex}
    Return the age of the Universe for a given z

\setlength{\parskip}{1ex}
    \end{boxedminipage}

    \label{pycosmicstar:cosmology:cosmology:setCosmologicalParameter}
    \index{pycosmicstar \textit{(package)}!pycosmicstar.cosmology \textit{(module)}!pycosmicstar.cosmology.cosmology \textit{(class)}!pycosmicstar.cosmology.cosmology.setCosmologicalParameter \textit{(method)}}

    \vspace{0.5ex}

\hspace{.8\funcindent}\begin{boxedminipage}{\funcwidth}

    \raggedright \textbf{setCosmologicalParameter}(\textit{self})

\setlength{\parskip}{2ex}
\setlength{\parskip}{1ex}
    \end{boxedminipage}

    \label{pycosmicstar:cosmology:cosmology:getCosmologicalParameter}
    \index{pycosmicstar \textit{(package)}!pycosmicstar.cosmology \textit{(module)}!pycosmicstar.cosmology.cosmology \textit{(class)}!pycosmicstar.cosmology.cosmology.getCosmologicalParameter \textit{(method)}}

    \vspace{0.5ex}

\hspace{.8\funcindent}\begin{boxedminipage}{\funcwidth}

    \raggedright \textbf{getCosmologicalParameter}(\textit{self})

\setlength{\parskip}{2ex}
\setlength{\parskip}{1ex}
    \end{boxedminipage}

    \label{pycosmicstar:cosmology:cosmology:getTilt}
    \index{pycosmicstar \textit{(package)}!pycosmicstar.cosmology \textit{(module)}!pycosmicstar.cosmology.cosmology \textit{(class)}!pycosmicstar.cosmology.cosmology.getTilt \textit{(method)}}

    \vspace{0.5ex}

\hspace{.8\funcindent}\begin{boxedminipage}{\funcwidth}

    \raggedright \textbf{getTilt}(\textit{self})

\setlength{\parskip}{2ex}
\setlength{\parskip}{1ex}
    \end{boxedminipage}

    \label{pycosmicstar:cosmology:cosmology:getRobr0}
    \index{pycosmicstar \textit{(package)}!pycosmicstar.cosmology \textit{(module)}!pycosmicstar.cosmology.cosmology \textit{(class)}!pycosmicstar.cosmology.cosmology.getRobr0 \textit{(method)}}

    \vspace{0.5ex}

\hspace{.8\funcindent}\begin{boxedminipage}{\funcwidth}

    \raggedright \textbf{getRobr0}(\textit{self})

\setlength{\parskip}{2ex}
\setlength{\parskip}{1ex}
    \end{boxedminipage}

    \label{pycosmicstar:cosmology:cosmology:getRodm0}
    \index{pycosmicstar \textit{(package)}!pycosmicstar.cosmology \textit{(module)}!pycosmicstar.cosmology.cosmology \textit{(class)}!pycosmicstar.cosmology.cosmology.getRodm0 \textit{(method)}}

    \vspace{0.5ex}

\hspace{.8\funcindent}\begin{boxedminipage}{\funcwidth}

    \raggedright \textbf{getRodm0}(\textit{self})

\setlength{\parskip}{2ex}
\setlength{\parskip}{1ex}
    \end{boxedminipage}

    \index{pycosmicstar \textit{(package)}!pycosmicstar.cosmology \textit{(module)}!pycosmicstar.cosmology.cosmology \textit{(class)}|)}
    \index{pycosmicstar \textit{(package)}!pycosmicstar.cosmology \textit{(module)}|)}

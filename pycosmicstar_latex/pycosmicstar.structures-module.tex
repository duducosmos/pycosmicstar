%
% API Documentation for API Documentation
% Module pycosmicstar.structures
%
% Generated by epydoc 3.0.1
% [Mon Aug 11 14:45:22 2014]
%

%%%%%%%%%%%%%%%%%%%%%%%%%%%%%%%%%%%%%%%%%%%%%%%%%%%%%%%%%%%%%%%%%%%%%%%%%%%
%%                          Module Description                           %%
%%%%%%%%%%%%%%%%%%%%%%%%%%%%%%%%%%%%%%%%%%%%%%%%%%%%%%%%%%%%%%%%%%%%%%%%%%%

    \index{pycosmicstar \textit{(package)}!pycosmicstar.structures \textit{(module)}|(}
\section{Module pycosmicstar.structures}

    \label{pycosmicstar:structures}
\textbf{Version:} 1.0.1



\textbf{Author:} Eduardo dos Santos Pereira



\textbf{License:} GPLV3




%%%%%%%%%%%%%%%%%%%%%%%%%%%%%%%%%%%%%%%%%%%%%%%%%%%%%%%%%%%%%%%%%%%%%%%%%%%
%%                               Variables                               %%
%%%%%%%%%%%%%%%%%%%%%%%%%%%%%%%%%%%%%%%%%%%%%%%%%%%%%%%%%%%%%%%%%%%%%%%%%%%

  \subsection{Variables}

    \vspace{-1cm}
\hspace{\varindent}\begin{longtable}{|p{\varnamewidth}|p{\vardescrwidth}|l}
\cline{1-2}
\cline{1-2} \centering \textbf{Name} & \centering \textbf{Description}& \\
\cline{1-2}
\endhead\cline{1-2}\multicolumn{3}{r}{\small\textit{continued on next page}}\\\endfoot\cline{1-2}
\endlastfoot\raggedright \_\-\_\-e\-m\-a\-i\-l\-\_\-\_\- & \raggedright \textbf{Value:} 
{\tt \texttt{'}\texttt{pereira.somoza@gmail.com}\texttt{'}}&\\
\cline{1-2}
\raggedright \_\-\_\-c\-r\-e\-d\-i\-t\-s\-\_\-\_\- & \raggedright \textbf{Value:} 
{\tt \texttt{[}\texttt{'}\texttt{Eduardo dos Santos Pereira}\texttt{'}\texttt{]}}&\\
\cline{1-2}
\raggedright \_\-\_\-m\-a\-i\-n\-t\-a\-i\-n\-e\-r\-\_\-\_\- & \raggedright \textbf{Value:} 
{\tt \texttt{'}\texttt{Eduardo dos Santos Pereira}\texttt{'}}&\\
\cline{1-2}
\raggedright \_\-\_\-s\-t\-a\-t\-u\-s\-\_\-\_\- & \raggedright Cosmological Dark Halos History From the formalism of Reed et al 
          (MNRAS, 346, 565-572, 2003) it is calculated the mass fraction of
          dark matter halos. The code obtain the mass density of dark halos
          and the fraction of brions into structures as a function of the 
          time. Here is used the Transfer function from Efstathiou, Bond \&
          White -- (MNRAS, 258, 1P, 1992). The current version it is 
          assumed the normalization of WMAP (withou gravitational waves) 
          adapted from Eisenstein e Hu (ApJ 511, 5, 1999) that in the way 
          that return  sigma\_8 = 0,84. The fraction of mass of dark halos 
          is obtained by the work of Sheth e Tormen (MNRAS 308, 119, 1999).
          All models consider Omega\_Total = Omega\_M + Omega\_L = 1,0

          This file is part of pystar. copyright : Eduardo dos Santos 
          Pereira

          pystar is free software: you can redistribute it and/or modify it
          under the terms of the GNU General Public License as published by
          the Free Software Foundation, either version 3 of the License. 
          pystar is distributed in the hope that it will be useful, but 
          WITHOUT ANY WARRANTY; without even the implied warranty of 
          MERCHANTABILITY or FITNESS FOR A PARTICULAR PURPOSE.  See the GNU
          General Public License for more details.

          You should have received a copy of the GNU General Public License
          along with Foobar.  If not, see 
          {\textless}http://www.gnu.org/licenses/{\textgreater}.

\textbf{Value:} 
{\tt \texttt{'}\texttt{Stable}\texttt{'}}&\\
\cline{1-2}
\raggedright p\-y\-v\-e\-r\-s\-i\-o\-n\- & \raggedright \textbf{Value:} 
{\tt sys.version\_info(major=2, minor=7, micro=6, releaselevel=\texttt{...}}&\\
\cline{1-2}
\raggedright \_\-\_\-p\-a\-c\-k\-a\-g\-e\-\_\-\_\- & \raggedright \textbf{Value:} 
{\tt \texttt{'}\texttt{pycosmicstar}\texttt{'}}&\\
\cline{1-2}
\end{longtable}


%%%%%%%%%%%%%%%%%%%%%%%%%%%%%%%%%%%%%%%%%%%%%%%%%%%%%%%%%%%%%%%%%%%%%%%%%%%
%%                           Class Description                           %%
%%%%%%%%%%%%%%%%%%%%%%%%%%%%%%%%%%%%%%%%%%%%%%%%%%%%%%%%%%%%%%%%%%%%%%%%%%%

    \index{pycosmicstar \textit{(package)}!pycosmicstar.structures \textit{(module)}!pycosmicstar.structures.structures \textit{(class)}|(}
\subsection{Class structures}

    \label{pycosmicstar:structures:structures}
\begin{tabular}{cccccc}
% Line for pycosmicstar.structuresabstract.structuresabstract, linespec=[False]
\multicolumn{2}{r}{\settowidth{\BCL}{pycosmicstar.structuresabstract.structuresabstract}\multirow{2}{\BCL}{pycosmicstar.structuresabstract.structuresabstract}}
&&
  \\\cline{3-3}
  &&\multicolumn{1}{c|}{}
&&
  \\
&&\multicolumn{2}{l}{\textbf{pycosmicstar.structures.structures}}
\end{tabular}

\textbf{Known Subclasses:} pycosmicstar.cosmicstarformation.cosmicstarformation

\begin{alltt}
This class was contructed based in the like Press-Schechter formalism
that provides characteristis like numerical density of dark matter halos
into the range m\_h, m\_h + dm\_h, the fraction of barionic matter,
and,  the accretion rate of barions into structures and the total number
of dark halos.

The models used to develop this class was presented for the first time
in the article of Pereira and Miranda (2010) - (MNRAS, 401, 1924, 2010).

The cosmologic background model is passed as a instance parameter:
    cosmology

Keyword arguments:
    lmin -- (default 6.0) log10 of the minal mass of the dark halo
                        where it is possible to have star formation.

    zmax -- (defaul 20.0) - the maximum redshift to be considered

    omegam -- (default 0.24) - The dark matter parameter

    omegab -- (default 0.04) - The barionic parameter

    omegal -- (default 0.73) - The dark energy parameter

    h -- (default 0.7) - The h of the Hubble constant (H = h * 100)

    massFunctionType:
        (Dark Haloes Mass Function)
        default 'ST' - Sheth et al. (2001) - z=[0,2]
        'TK' - Tinker et al. (2008) - z=[0,2.5]
        'PS' - Press and Schechter (1974) - z=-
        'JK' - Jenkins et al. (2001) z=[0,5]
        'W' - Warren et al. (2006) z=0
        'WT1' - Watson et al. (2013) - Tinker Modified - z=[0,30]
        'WT2' - Watson et al. (2013) - Gamma times times Tinker Modified
                                            z=[0,30]
        'B' - Burr Distribuction. Marassi and Lima (2006) - Press Schechter
                                modified.
    qBurr:
        (default 1) - The q value of Burr Distribuction.
\end{alltt}


%%%%%%%%%%%%%%%%%%%%%%%%%%%%%%%%%%%%%%%%%%%%%%%%%%%%%%%%%%%%%%%%%%%%%%%%%%%
%%                                Methods                                %%
%%%%%%%%%%%%%%%%%%%%%%%%%%%%%%%%%%%%%%%%%%%%%%%%%%%%%%%%%%%%%%%%%%%%%%%%%%%

  \subsubsection{Methods}

    \label{pycosmicstar:structures:structures:__init__}
    \index{pycosmicstar \textit{(package)}!pycosmicstar.structures \textit{(module)}!pycosmicstar.structures.structures \textit{(class)}!pycosmicstar.structures.structures.\_\_init\_\_ \textit{(method)}}

    \vspace{0.5ex}

\hspace{.8\funcindent}\begin{boxedminipage}{\funcwidth}

    \raggedright \textbf{\_\_init\_\_}(\textit{self}, \textit{cosmology}, **\textit{kwargs})

\setlength{\parskip}{2ex}
\setlength{\parskip}{1ex}
    \end{boxedminipage}

    \vspace{0.5ex}

\hspace{.8\funcindent}\begin{boxedminipage}{\funcwidth}

    \raggedright \textbf{abt}(\textit{self}, \textit{a})

    \vspace{-1.5ex}

    \rule{\textwidth}{0.5\fboxrule}
\setlength{\parskip}{2ex}
\begin{alltt}
Return the accretion rate of barionic matter, as
a function of scala factor, into strutures.

Keyword arguments:
    a -- scala factor (1.0 / (1.0 + z))
\end{alltt}

\setlength{\parskip}{1ex}
      Overrides: pycosmicstar.structuresabstract.structuresabstract.abt

    \end{boxedminipage}

    \vspace{0.5ex}

\hspace{.8\funcindent}\begin{boxedminipage}{\funcwidth}

    \raggedright \textbf{fbstruc}(\textit{self}, \textit{z})

    \vspace{-1.5ex}

    \rule{\textwidth}{0.5\fboxrule}
\setlength{\parskip}{2ex}
\begin{alltt}
Return the faction of barions into structures

Keyword arguments:
    z -- redshift
\end{alltt}

\setlength{\parskip}{1ex}
      Overrides: pycosmicstar.structuresabstract.structuresabstract.fbstruc

    \end{boxedminipage}

    \vspace{0.5ex}

\hspace{.8\funcindent}\begin{boxedminipage}{\funcwidth}

    \raggedright \textbf{fstm}(\textit{self}, \textit{lm})

    \vspace{-1.5ex}

    \rule{\textwidth}{0.5\fboxrule}
\setlength{\parskip}{2ex}
\begin{alltt}
Numerical function that return the value of sigm that
will be used by dfridr to calculate d\_sigma\_dlog10(m).

Keyword arguments:
    lm -- log10 of the mass of dark halo
\end{alltt}

\setlength{\parskip}{1ex}
      Overrides: pycosmicstar.structuresabstract.structuresabstract.fstm

    \end{boxedminipage}

    \vspace{0.5ex}

\hspace{.8\funcindent}\begin{boxedminipage}{\funcwidth}

    \raggedright \textbf{getCacheDir}(\textit{self})

    \vspace{-1.5ex}

    \rule{\textwidth}{0.5\fboxrule}
\setlength{\parskip}{2ex}
    Return True and cache name if the cache directory existe and false 
    else.

\setlength{\parskip}{1ex}
      Overrides: pycosmicstar.structuresabstract.structuresabstract.getCacheDir

    \end{boxedminipage}

    \label{pycosmicstar:structures:structures:getDeltaHTinker}
    \index{pycosmicstar \textit{(package)}!pycosmicstar.structures \textit{(module)}!pycosmicstar.structures.structures \textit{(class)}!pycosmicstar.structures.structures.getDeltaHTinker \textit{(method)}}

    \vspace{0.5ex}

\hspace{.8\funcindent}\begin{boxedminipage}{\funcwidth}

    \raggedright \textbf{getDeltaHTinker}(\textit{self})

\setlength{\parskip}{2ex}
\setlength{\parskip}{1ex}
    \end{boxedminipage}

    \label{pycosmicstar:structures:structures:getmassFunctionDict}
    \index{pycosmicstar \textit{(package)}!pycosmicstar.structures \textit{(module)}!pycosmicstar.structures.structures \textit{(class)}!pycosmicstar.structures.structures.getmassFunctionDict \textit{(method)}}

    \vspace{0.5ex}

\hspace{.8\funcindent}\begin{boxedminipage}{\funcwidth}

    \raggedright \textbf{getmassFunctionDict}(\textit{self})

    \vspace{-1.5ex}

    \rule{\textwidth}{0.5\fboxrule}
\setlength{\parskip}{2ex}
    Return a list with key and function of implemented dark haloes mass 
    function

\setlength{\parskip}{1ex}
    \end{boxedminipage}

    \vspace{0.5ex}

\hspace{.8\funcindent}\begin{boxedminipage}{\funcwidth}

    \raggedright \textbf{halos\_n}(\textit{self}, \textit{z})

    \vspace{-1.5ex}

    \rule{\textwidth}{0.5\fboxrule}
\setlength{\parskip}{2ex}
\begin{alltt}
Return the integral of the mass function of dark halos multiplied
by mass in the range of log(M\_min) a log(M\_max)

Keyword arguments:
    z -- redshift
\end{alltt}

\setlength{\parskip}{1ex}
      Overrides: pycosmicstar.structuresabstract.structuresabstract.halos\_n

    \end{boxedminipage}

    \vspace{0.5ex}

\hspace{.8\funcindent}\begin{boxedminipage}{\funcwidth}

    \raggedright \textbf{massFunction}(\textit{self}, \textit{lm}, \textit{z})

    \vspace{-1.5ex}

    \rule{\textwidth}{0.5\fboxrule}
\setlength{\parskip}{2ex}
\begin{alltt}
Return the mass function of dark halos.

Keyword arguments:
    lm -- log10 of the mass of the dark halo
    z -- redshift
\end{alltt}

\setlength{\parskip}{1ex}
      Overrides: pycosmicstar.structuresabstract.structuresabstract.massFunction

    \end{boxedminipage}

    \vspace{0.5ex}

\hspace{.8\funcindent}\begin{boxedminipage}{\funcwidth}

    \raggedright \textbf{numerical\_density\_halos}(\textit{self}, \textit{z})

    \vspace{-1.5ex}

    \rule{\textwidth}{0.5\fboxrule}
\setlength{\parskip}{2ex}
\begin{alltt}
Return the numerial density of dark halos
within the comove volume

Keyword arguments:
    z- redshift
\end{alltt}

\setlength{\parskip}{1ex}
      Overrides: pycosmicstar.structuresabstract.structuresabstract.numerical\_density\_halos

    \end{boxedminipage}

    \label{pycosmicstar:structures:structures:setDeltaHTinker}
    \index{pycosmicstar \textit{(package)}!pycosmicstar.structures \textit{(module)}!pycosmicstar.structures.structures \textit{(class)}!pycosmicstar.structures.structures.setDeltaHTinker \textit{(method)}}

    \vspace{0.5ex}

\hspace{.8\funcindent}\begin{boxedminipage}{\funcwidth}

    \raggedright \textbf{setDeltaHTinker}(\textit{self}, \textit{delta\_halo})

\setlength{\parskip}{2ex}
\setlength{\parskip}{1ex}
    \end{boxedminipage}

    \label{pycosmicstar:structures:structures:setMassFunctionDict}
    \index{pycosmicstar \textit{(package)}!pycosmicstar.structures \textit{(module)}!pycosmicstar.structures.structures \textit{(class)}!pycosmicstar.structures.structures.setMassFunctionDict \textit{(method)}}

    \vspace{0.5ex}

\hspace{.8\funcindent}\begin{boxedminipage}{\funcwidth}

    \raggedright \textbf{setMassFunctionDict}(\textit{self}, \textit{key}, \textit{function})

    \vspace{-1.5ex}

    \rule{\textwidth}{0.5\fboxrule}
\setlength{\parskip}{2ex}
    Add a new key and function in the dark haloes mass function dictionary

\setlength{\parskip}{1ex}
    \end{boxedminipage}

    \label{pycosmicstar:structures:structures:setQBurrFunction}
    \index{pycosmicstar \textit{(package)}!pycosmicstar.structures \textit{(module)}!pycosmicstar.structures.structures \textit{(class)}!pycosmicstar.structures.structures.setQBurrFunction \textit{(method)}}

    \vspace{0.5ex}

\hspace{.8\funcindent}\begin{boxedminipage}{\funcwidth}

    \raggedright \textbf{setQBurrFunction}(\textit{self}, \textit{q})

    \vspace{-1.5ex}

    \rule{\textwidth}{0.5\fboxrule}
\setlength{\parskip}{2ex}
    Set the q value of dark haloes mass function derived from Burr 
    distribuction.

\setlength{\parskip}{1ex}
    \end{boxedminipage}


\large{\textbf{\textit{Inherited from pycosmicstar.structuresabstract.structuresabstract\textit{(Section \ref{pycosmicstar:structuresabstract:structuresabstract})}}}}

\begin{quote}
cosmicStarFormationRate(), gasDensityInStructures()
\end{quote}
    \index{pycosmicstar \textit{(package)}!pycosmicstar.structures \textit{(module)}!pycosmicstar.structures.structures \textit{(class)}|)}
    \index{pycosmicstar \textit{(package)}!pycosmicstar.structures \textit{(module)}|)}
